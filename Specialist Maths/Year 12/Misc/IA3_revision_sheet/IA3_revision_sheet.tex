\documentclass[10pt,a4paper]{article}
\usepackage[T1]{fontenc}
\usepackage{fancyhdr}
\usepackage{pgfplots}
\usepackage{tikz}
\usepackage{bigints}
\usepackage[dvipsnames]{xcolor}
\usepackage{graphicx}
\usepackage{geometry}
\geometry{
	textwidth=17cm,
	headheight=10mm,
	bottom=2cm}
\newcommand{\dotline}{\noindent\makebox[\linewidth]{\dotfill}\\[1em]}
	
	
\begin{document}
	\begin{titlepage}
		\title{ \textbf{\Huge Speshlist Mathematics \\ \hfill \\ \hfill \\ IA3 Exam Worked Examples}}
		\date{\hfill	\\ \hfill \\ \hfill \\ \hfill	\includegraphics[width=0.8\paperwidth]{cat.png}}

	\end{titlepage}
	\clearpage
	\maketitle
	\thispagestyle{empty}

	\newpage
	\section{Geometric Proofs using vectors}
	\pagestyle{fancy}
	\fancyhead[R]{Speshlist Mathematics}
	\fancyhead[C]{Geometric Proofs using vectors}
	\fancyfoot[L]{Cambridge - Chapter 4}
	\fancyfoot[C]{Vectors}
	\fancyfoot[R]{Page \thepage \;of \PreviousTotalPages}
	 
\textbf{\textcolor{OliveGreen}{Example 4.5}:}\\
Prove the diagonals of a parallelogram meet ar right angles if and only if it is a rhombus.\\

\dotline
\dotline
\dotline
\dotline
\dotline
\dotline
\dotline
\dotline
\dotline
\dotline
\dotline
\dotline
\dotline
\dotline
\dotline
\dotline
\dotline
\dotline
\dotline
\dotline
\dotline
\dotline
\dotline
\dotline
\dotline
\dotline
\dotline
\dotline



\section{Integration and applications of integration}

\fancyhead[R]{Speshlist Mathematics}
\fancyhead[C]{Integration and applications of integration}
\fancyfoot[L]{Cambridge - Chapter 11}
\fancyfoot[C]{Vectors}
\fancyfoot[R]{Page \thepage \;of \PreviousTotalPages}



\textbf{\textcolor{OliveGreen}{Example 11.14}:}\\
Evaluate the following:\\
(a)\;$\bigintss \cos ^2 x \;dx$\hfill(c)\;$\bigintss \sin(2x)\cos(2x)dx$\hfill(e)\;$\bigintss \sin^3x \;\cos ^2x\;dx$\\
\\
(b)\;$\bigintss \tan^2 x \; dx$\hfill(d)$\bigintss \cos^4 x \;dx\hfill\hfill\;\;\;\;\;\;\;\;\;\;\;$\\
\\

\dotline
\dotline
\dotline
\dotline
\dotline
\dotline
\dotline
\dotline
\dotline
\dotline
\dotline
\dotline
\dotline
\dotline
\dotline
\dotline
\dotline
\dotline
\dotline
\dotline
\dotline
\dotline
\dotline
\dotline
\dotline
\newpage



\fancyhead[R]{Speshlist Mathematics}
\fancyhead[C]{Integration and applications of integration}
\fancyfoot[L]{Cambridge - Chapter 11}
\fancyfoot[C]{Vectors}
\fancyfoot[R]{Page \thepage \;of \PreviousTotalPages}


\noindent
\textbf{\textcolor{OliveGreen}{Example 11.24}:}\\
Determine $\bigintss{ x^2 e^x }dx$.\\

\dotline
\dotline
\dotline
\dotline
\dotline
\dotline
\dotline
\dotline
\dotline
\dotline
\dotline
\dotline
\dotline
\dotline
\dotline
\dotline
\dotline
\dotline
\dotline
\dotline
\dotline
\dotline
\dotline
\dotline
\dotline
\dotline
\dotline





\section{Vector calculus}

\fancyhead[R]{Speshlist Mathematics}
\fancyhead[C]{Vector calculus}
\fancyfoot[L]{Cambridge - Chapter 8}
\fancyfoot[C]{Vector calculus}
\fancyfoot[R]{Page \thepage \;of \PreviousTotalPages}



\textbf{\textcolor{OliveGreen}{Example 8.19}:}\\
For position vectors at time $t\geq 0$, particles $A$ and $B$ are given by 
\begin{align*}
	\boldsymbol{r}_A (t)& = (t^3 - 9t + 8)\boldsymbol{i}+t^2\boldsymbol{j}\\
	\boldsymbol{r}_B (t)& = (2-t^2)\boldsymbol{i} + (3t-2)\boldsymbol{j}
\end{align*}
Prove that $A$ and $B$ will collide while travelling at the same speed but at right angles to each other.

\dotline
\dotline
\dotline
\dotline
\dotline
\dotline
\dotline
\dotline
\dotline
\dotline
\dotline
\dotline
\dotline
\dotline
\dotline
\dotline
\dotline
\dotline
\dotline
\dotline
\dotline
\dotline
\dotline
\dotline
\dotline
\dotline




\end{document}