\documentclass[11pt, letterpaper]{article}

\begin{document}

\title{Modeling and Lighting Interior Spaces using Reflected Natural Light}
\author{Noah Alexiou}
\date{\today}
\maketitle
\newpage
\tableofcontents
\newpage


\section{Introduction}


\subsection{Premise}
- Study aims to find optimal mirror placement in order to light as much of the cave as possible




\subsection{Assumptions}
\par
In Order for a solution to be formed, a set of constants must be assumed.
\\
<<<<<<< HEAD
- The walls of the cave are perfectly parallel to the floor of the cave and extend upwards with an undefined height. This simplifies modeling and can easily be altered to fit revised specifications.
\\
- 
\\
- Light does not decrease in intensity as it travels through the cave and bounces off mirrors
\\
- Light will enter the cave on the same plane as the mirrors and reflect across their entire length
\\


\subsection{Observations}
\par

- Mirrors will stand parallel from the ground and extend to the ceiling, Dimensions unknown... Possibly variable
\\
- Light can be modeled as a relative position vector with origin at a mirror surface
\\
- 

=======
- Light brightness does not decrease as it travels through the cave and bounces off mirrors
\\
-

\subsection{Observations}
\par
- Light can be modelled as a vector
\\
- Mirrors will stand parralel from the ground and entend to the ceiling, Dimentions unknown... Possibly variable
\\
- Light will reflect so that $angle \: of \:incidence = angle \: of \: refraction$, or 
>>>>>>> 3928289fa94fcd2bfff4712c73cce29c0c33681e

\subsection{Translation}


\section{Solve}


\section{Evaluate}



\subsection{Reasonableness}


\subsection{Strenghts and Limitations of Solution}


\section{Conclusion}


 

\end{document}
