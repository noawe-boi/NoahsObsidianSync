
\documentclass[11pt, letterpaper]{article}
%\usepackage{bookmark}
\usepackage[a4paper,margin=2cm]{geometry}
\usepackage{bm}
\begin{document}

\title{Modeling and Lighting Interior Spaces using Reflected Natural Light}
\author{Noah Alexiou}
\date{\today}
\maketitle
\newpage
\tableofcontents
\newpage


\section{Introduction}


\subsection{Premise}
- Study aims to find optimal mirror placement for minimum decrease in intensity as light travels throughout the cave




\subsection{Assumptions}
\par
In Order for a solution to be formed, a set of constants must be assumed.
\begin{itemize}
	\item The mirrors and walls of the cave are perfectly parallel to the floor of the cave and extend upwards with an undefined height. This simplifies modeling and can easily be altered to fit revised specifications, such as only needing to light the floor of the cave.
	
	\item Either light does not decrease in intensity or increase in area according to the inverse square law, or the change is negligible. The suns rays have traveled so far that they can be considered effectively parallel and therefore will not diverge. 
	\textbf{find a source for this}
		
	\item Light will enter the cave parallel to the ground and of equal intensity from floor to ceiling.
	
	\item 
\end{itemize}

\subsection{Observations}
\par
\begin{itemize}
\item Vectors can be modeled on the Cartesian plane as 

\item Light can be modeled as a relative position vector with origin at a mirror surface

\item Light will reflect so that $\textrm{angle  of incidence} = \textrm{angle of refraction}$. This relationship can be modeled as the vectors $\bm{i}$ co-ordinate being scaled by a factor of $-1$ when it reflects. i.e. a reflection on the $x$-axis.

\item Light will not diverge however contaminatns in the air may decrease the intensity of light. Distance light travels in cave must me minimized

\item  Since it is assumed that light will not diverge. the maximum size a mirror must be to reflect all the light hitting it will be 2 units, or the size of the cave's entrance.
\end{itemize}





\subsection{Translation}
\par 

\begin{itemize}
\item Vector addition can be used to join each given vector and form the walls of the cave and the obstacles

\item Used Excel to do vector addition

\item Took points from excel and inserted them as separate $x$ and $y$ list in desmos

\item Graphed each point on list and joined points with lines.
\end{itemize}


\par 


\section{Solve}

\par 


\section{Evaluate}



\subsection{Reasonableness}


\subsection{Strengths and Limitations of Solution}


\section{Conclusion}


 

\end{document}
