
\documentclass[11pt, letterpaper]{article}
\begin{document}

\title{Modeling and Lighting Interior Spaces using Reflected Natural Light}
\author{Noah Alexiou}
\date{\today}
\maketitle
\newpage
\tableofcontents
\newpage


\section{Introduction}


\subsection{Premise}
- Study aims to find optimal mirror placement in order to light as much of the cave as possible




\subsection{Assumptions}
\par
In Order for a solution to be formed, a set of constants must be assumed.
\\
- The walls of the cave are perfectly parallel to the floor of the cave and extend upwards with an undefined height. This simplifies modeling and can easily be altered to fit revised specifications.
\\
- Light does not decrease in intensity as it travels through the cave and bounces off mirrors
\\
- Light will enter the cave parallel to the ground and reflect across the entire length of mirrors


\subsection{Observations}
\par

- Light can be modeled as a relative position vector with origin at a mirror surface
\\
- Light will reflect so that $angle \: of \:incidence = angle \: of \: refraction$, or 
\\
- 




\subsection{Translation}


\section{Solve}


\section{Evaluate}



\subsection{Reasonableness}


\subsection{Strenghts and Limitations of Solution}


\section{Conclusion}


 

\end{document}
