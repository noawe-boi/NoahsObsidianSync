 \documentclass[12pt,a4paper]{article}
\usepackage[a4paper,margin=2cm]{geometry}
\usepackage[]{graphicx}
\usepackage{bm}
\usepackage[strings]{underscore}
\usepackage{apacite}
\setlength{\parindent}{0pt}
\graphicspath{ {./Images} }
\usepackage{wrapfig}
\usepackage[autostyle, english = american]{csquotes}
\usepackage[toc,page]{appendix}
\usepackage{amssymb}
\MakeOuterQuote{"}
\usepackage[T1]{fontenc}

\begin{document}
\begin{titlepage}


\title{Determining the relationship between hanging masses and the angle of a frictionless plane}

\author{Noah Alexiou}


\date{April 2025}

\maketitle
\centering

\end{titlepage}
\tableofcontents
\newpage
\section{Preface}
Hey Sir,
Yeah I know this doesn't look good.
Tbh even if I did submit yesterday it would basically be a blank document still. 
Not really much to say about this tbh, so I wouldn't be offended if you literally just didnt mark my draft considering that it is late.
\section{Introduction}

\subsection{Research Question}
When the mass of an object on a frictionless plane is altered, and the mass of a hanging object adjusted so  equilibrium is achieved, can this be used to measure the angle of the plane? What is the accuracy of this method in comparison to conventional measuring techniques. 

\subsection{Rationale}

\subsubsection{Hypothesis}

\subsection{Methodology}

\subsubsection{Modifications}

\subsubsection{Materials}
\begin{itemize}
	\item Angle gun 
	\item Frictionless plane
	\item Brass weights
	\item Blue tack 
	\item Scale
	\item Carriage
\end{itemize}

\subsubsection{Method}
\begin{enumerate}
\item Set up slope at a constant angle. It will remain at this angle for the entire duration of the experiment. 
\item Set the hanging mass for the respective set of trials $(h)$. 
\item Alter the cart mass $(c)$ until equilibrium with $h$ is achieved.
\item Measure $h$ and record masses. 
\item Repeat for set number of trials and $h$ values.
	
	 
\end{enumerate}


\subsubsection{Risk Assessment}
Frictionless plane
\begin{itemize}
	\item Mishandling of heavy masses on the frictionless plane could result in them sliding down the slope at high speed. This could damage equipment of cause injury. The slope will be turned off not required, and one person will always be supporting the cart whenever possible to prevent this. 
	\item Using too low fan speed on the frictionless plane may not create enough of an air pocket to support heavy weights. This could cause rubbing between the surfaces which could damage both the plane and carriage. The plane will be set to the highest possible speed throughout the experiment to negate the possibility of this occurring.
\end{itemize}
Masses
\begin{itemize}
	\item Heavy masses or items containing many brass weights may cause injury if dropped or mishandled. participants will wear enclosed footwear to negate injury if this occurs.  
\end{itemize}
\section{Results and Evaluation}
\subsection{Results}


Data was plotted in excel, with $c$ on the $x$-axis, and  $h$ on the $y$-axis. A trendline was formed from the graph. f

In theory, the data should have represented the equation:
$$c=h\frac{1}{\sin(\theta)}$$
which was rearranged to give 
$$\frac{h}{c}=\sin{(\theta)}$$
$$\frac{c}{h}=\mathrm{gradient}$$
$$\therefore \frac{h}{c}=\frac{1}{\mathrm{gradient}}$$
$$\therefore \sin(\theta)=\frac{1}{\textrm{gradient}}$$
$$\therefore \sin^{-1}\left(\frac{1}{
\textrm{gradient}}\right)=\theta$$
\subsection{Discussion}


\section{Conclusion}

	
	
	
\end{document}