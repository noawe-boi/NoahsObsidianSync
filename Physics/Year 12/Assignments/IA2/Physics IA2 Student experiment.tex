\documentclass[11pt,a4paper]{article}
\usepackage[a4paper,margin=2cm]{geometry}
\usepackage[]{graphicx}
\usepackage{bm}
\usepackage[strings]{underscore}
\usepackage{apacite}
\setlength{\parindent}{0pt}
\graphicspath{ {./Images} }
\usepackage{wrapfig}
\usepackage[autostyle, english = american]{csquotes}
\usepackage[toc,page]{appendix}
\usepackage{amssymb}
\MakeOuterQuote{"}
\usepackage[T1]{fontenc}

\begin{document}
\begin{titlepage}


\title{Determining the relationship between hanging masses and the angle of a frictionless plane}

\author{Noah Alexiou}


\date{April 2025}

\maketitle
\centering

\end{titlepage}
\tableofcontents
\newpage

\section{Introduction}

\subsection{Research Question}
When the mass of an object on a frictionless plane is altered, and the mass of a hanging object adjusted so  equilibrium is achieved, Can this be used to find the angle of the plane?

\subsection{Rationale}

\subsubsection{Hypothesis}

\subsection{Methodology}

\subsubsection{Modifications}

\subsubsection{Materials}
\begin{itemize}
	\item Angle gun 
	\item Frictionless plane
	\item Brass weights
	\item Blue tack 
	\item Scale
	\item Carriage
\end{itemize}

\subsubsection{Method}
\begin{itemize}
	\item Set up slope at angle that is to be measured.
	\item Measure angle of the slope using angle gun
	\item Place the first mass on the carriage
	\item Choose a reasonable starting mass for the hanging mass
	\item Engage the frictionless slope and alter hanging mass by adding or removing brass weights or blue tack until both masses are in equilibrium
	\item Record masses
	\item Repeat for each carriage weight and slope angle
	\item Perform Calculations
	
	 
\end{itemize}


\subsubsection{Risk Assessment}

\section{Results and Evaluation}
\subsection{Results}
\subsection{Discussion}


\section{Conclusion}

	
	
	
\end{document}