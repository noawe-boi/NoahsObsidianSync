\documentclass[12pt,a4paper]{article}
\usepackage[a4paper,margin=2cm]{geometry}
\usepackage[]{graphicx}
\usepackage{bm}
\usepackage[strings]{underscore}
\usepackage{apacite}
\setlength{\parindent}{0pt}
\graphicspath{ {./Images} }
\usepackage{wrapfig}
\usepackage[autostyle, english = american]{csquotes}
\usepackage[toc,page]{appendix}
\usepackage{amssymb}
\MakeOuterQuote{"}
\usepackage[T1]{fontenc}

\begin{document}
\begin{titlepage}


\title{Determining the relationship between hanging masses and the angle of a frictionless plane}

\author{Noah Alexiou}


\date{April 2025}

\maketitle
\centering

\end{titlepage}
\tableofcontents
\newpage

\section{Introduction}

\subsection{Research Question}
When the mass of an object on a frictionless plane is altered, and the mass of a hanging object adjusted so  equilibrium is achieved, Can this be used to find the angle of the plane?

\subsection{Rationale}

\subsubsection{Hypothesis}

\subsection{Methodology}

\subsubsection{Modifications}

\subsubsection{Materials}
\begin{itemize}
	\item Angle gun 
	\item Frictionless plane
	\item Brass weights
	\item Blue tack 
	\item Scale
	\item Carriage
\end{itemize}

\subsubsection{Method}
\begin{enumerate}
\item Set up slope at a constant angle. It will remain at this angle for the entire duration of the experiment. 
\item Set the hanging mass for the respective set of trials $(m_2)$. 
\item Alter the cart mass $(m_1)$ until equilibrium with $m_2$ is achieved.
\item Measure $m_1$ and $m_2$ 
\item Repeat for set number of trials and $m_2$ values.
	
	 
\end{enumerate}


\subsubsection{Risk Assessment}

\section{Results and Evaluation}
\subsection{Results}
\subsection{Discussion}


\section{Conclusion}

	
	
	
\end{document}